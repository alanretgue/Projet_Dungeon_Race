\documentclass[12pt]{report}
\title{\LARGE Studio Run \\ \large Dungeon Race}
\date{}
\author{\small Thibault THOMAS \\ \small Thomas PHILIP \\ \small Alan GUERET \\ \small Théo DARONAT}
\begin{document}
\maketitle
\tableofcontents
\pagebreak
\section{Introduction}
Il faut rédiger une introduction d'au minimum une page, c'est la merde. Après les pages ont l'air d'être petite. Les gars, c'est le feu. Une ligne sur TeXmaker équivaut à une ligne sur le doc pdf ! Ah, nan finalement...
\section{Plan du projet}

	\subsection{Origine de l'étude}
	La question de la nature du projet ne se posait même pas au sein du groupe, nous voulions faire un jeu. Nous devions travailler sur un projet en équipe et apprendre de nouvelles choses tout en y prenant plaisir, et c’est le jeu vidéo qui nous parlait le plus. C’est ainsi que nous avons pris la décision d’en réaliser un. Après il ne restait plus qu’à décider quel type de jeu nous voulions réaliser. Au début, Alan avait proposé de réaliser un RPG (Role Play Game), mais nous avons plus opté pour l’idée de Thibault : la DeathRun. Mais une DeathRun simple ne suffisait pas, nous voulions ajouter une touche plus personnelle sur le mode, alors nous nous sommes dit que nous allions mélanger le mode de jeu de la DeathRun et du RPG : il en ressort une DeathRun avec différents personnages qui possèdent chacun des attributs spécifiques et des bonus.
	
	\subsection{Objet de l'étude}
		Ce projet d’informatique possède plusieurs avantages. Tout d’abord nous allons travailler en équipe. En effet, nous avons un chef de projet et chacun est responsable ou suppléant dans un domaine de réalisation. Cela va nous permettre d'acquérir une expérience indispensable en tant que futurs ingénieurs. De plus, nous pourrons aborder de nouvelles notions tel que le développement réseau (site web, réseau multi-joueurs…) ou encore le modélisme, que nous n’aurions pas forcément vus en classe. Nous aurons aussi l’occasion de travailler sur de la POO (Programmation Orientée Objet), assez différente de la programmation fonctionnelle que nous avons majoritairement vue en cours jusque là, et qui est pourtant indispensable dans le domaine de la programmation.
\pagebreak

Avis de Thibault: \\
Personnellement, en arrivant dans cette école, je n’avais aucune notion en programmation. Ce projet est une aubaine pour moi, c’est l’occasion de combler mon domaine de compétence. C’est d’ailleurs pour ça que je me suis proposé pour être responsable du développement du site internet, cette compétence est basique est indispensable en tant qu’informaticien. De plus, j’aurai l’occasion de travailler sur le code même du jeu (surtout de la POO à vrai dire) en tant que suppléant. Cette notion étant assez complexe à appréhender en tant que débutant, j’aurai l’occasion de me plonger pleinement dedans et ainsi de progresser un maximum. \\

Avis d'Alan:\\

Avis de Thomas:\\

Avis de Théo:\\

	\subsection{\'Etat de l'art}
	\paragraph{}
	Le type de jeu Death Run est en réalité un mode de jeu incorporé dans des jeux déjà existant. Le premier à avoir eu ce mode de jeu est le "jeu" Garry's Mod qui est un jeu "sandbox" (jeu bac à sable) où le principe est de créer des modes de jeu. Ce sont les utilisateurs qui font évoluer le jeu. Ce dernier, créé par Garry Newman, a été publié par la société Valve en 2006, il faudra attendre 2009 pour que le mode de jeu Death Run se popularise. Le principe du mode Death Run est plutôt simple, il y a deux équipes. La première est celle des coureurs, leur but est d'aller d'un bout à l'autre de la map, la seconde est composé généralement d'un seul joueur qui a pour objectif d'empêcher les coureurs de terminer la course, en activant des pièges. Ce mode de jeu possède plusieurs aspects attrayant, comme la compétitivité par exemple. En effet pour progresser dans ce mode, on doit pratiquer et acquérir des compétences de mécanique de jeu qui n'ont aucune limite : on a toujours une marge de progrès. De plus l'aspect teamplay et coopératif en multijoueur rend le jeu toujours plus amusant. Enfin, l'activité constante de la communauté pemret un renouvellement constant du mode (maps, pièges, etc...), ce qui règle le problème de la lassitude à long terme. En effet, les possibilités de création et d'innovation sont immenses : la seule limite est l'imagination.
	\paragraph{}
	On peut évoquer d'autres jeux populaires qui utilisent le mode de la Death Run, comme Counter-Strike. Ce dernier est à la base un mode de jeu de Half Life, un fps sorti en 1998 et développé par Valve. Il est sorti en tant que jeu officiel en 2000. Ce qui nous intéresse ici n'est pas tant le mode principal de Counter-Strike, mais plutôt l'aspect communautaire. En effet, Valve a mis à disposition des moyens de créer des serveurs publiques par et pour la communauté, et c'est ainsi qu'est né le DeathRun sur Counter-Strike.
	Il existe d'autres jeux populaires possédant ce mode jeu tel que Team Fortress 2 : un jeu de tir à la première personne multijoueur en ligne Développé par Valve Corporation qui est construit autour de deux équipes s'affrontant pour un objectif. On peut également cité le jeu le plus populaire depuis un an qui n'est nulle autre que Fornite, un jeu style Battle Royal possédant des pièges et un mode bac à sable, ce qui a permis la création d'un mode Death Run. Enfin, on peut évoquer le jeu Minecraft qui, grâce à la grande liberté qu'il nous offre, permet aux joueurs de créer de nombreux modes de jeu dont des Death Run.
	
	\subsection{Description et mécanique du jeu}
	
		mettre le gameplay et tout ce qu'on a eu comme idée pour le jeu
	
	\subsection{Découpage du projet} % technologique et méthodologique
		
	Multi: \\
Ceux qui s'occupent de cette partie devront implémenter en C\# le 
fait de pouvoir jouer à notre jeu à plusieurs. Pour se faire nous
utiliserons photon qui est une librairie C\#. \\
\\ Site Internet:\\
Pour créer un site internet nous utiliserons un framework tel que
ruby on rail ou angular js. Nous y présenterons et mettrons à
disposition notre jeu. Nous nous aiderons d'autres sites web et 
de cours pour pouvoir le réaliser.\\
\\ Graphisme et Son:\\
Les personnes qui s'occupent de cette partie s'occuperont de
toute la partie visuelle du jeu. Ils utiliseront blender pour la
la partie visuelle. Pour la parie audio, ils utiliseront des
des instruments diverses (guitare, batterie...) ou simplement
leurs voies. Ils utiliseront aussi des logiciels de montage.\\
\\ Code Unity:\\
Cette partie a pour but de réaliser toute la partie logique du
jeu. En effet, ceux qui feront cette partie seront en charge
d'implémenter la physique du jeu, les menus, les gestion de 
tous les éléments qui doivent apparaître à l'écrans... Nous
utiliserons le C\# avec la plateforme Unity.\\
	
	\begin{tabular}{|*{5}{c|}}
	\hline
		& Réseaux - Multi & Site internet & Graphisme et son & Code unity \\
		\hline
		Thomas PHILIP & - & - & Suppléant & Responsable \\
		\hline
		Alan GUERET & Responsable & Suppléant & - & - \\
		\hline
		Thibault THOMAS & - & Responsable & - & Suppléant \\
		\hline
		Théo DARONAT & Suppléant & - & Responsable & - \\
		\hline
	
	\end{tabular}
	\\
	\\
	
	\begin{tabular}{|*{4}{c|}}
	\hline
	& \multicolumn{3}{|c|}{Deadline}\\
	\hline
	Tâche & du 9 au 13 mars & du 20 au 24 avril & du 28 mai au 10 juin \\
	\hline
	Interface jeu & & x & \\
	\hline
	Map multi (modélisation) & & x & \\
	\hline
	synchronisation multi & x & & \\
	\hline
	Map solo (modélistion) & & x & \\
	\hline
	effet sonore & & & x \\
	\hline
	ambiance sonore & & & x \\
	\hline
	site web & x & & \\
	\hline 
	physique des objets & x & & \\
	\hline 
	level design & & & x \\
	\hline 
	IA des ennemis & & x & \\
	\hline
	IA du boss & & & x \\
	\hline
	
	\end{tabular}
\section{Conclusion}
\end{document}