\documentclass[12pt]{report}
\usepackage{url}
\title{\LARGE Cahier des charges \\ \large Studio Run : Dungeon Race}
\date{}
\author{\small Thibault THOMAS \\ \small Thomas PHILIP \\ \small Alan GUERET \\ \small Théo DARONAT}
\begin{document}
\maketitle
\tableofcontents

\pagebreak
\chapter{Introduction}
\paragraph{}
Il faut rédiger une introduction d'au minimum une page, c'est la merde. Après les pages ont l'air d'être petite. Les gars, c'est le feu. Une ligne sur TeXmaker équivaut à une ligne sur le doc pdf ! Ah, nan finalement...
\chapter{Le Groupe}
	
	\section{Création du groupe}
	\paragraph{}
	Notre équipe est composée de quatre personnes : Thomas Philip, Théo Daronat, Alan Gueret et Thibault Thomas. 
Thibault, Théo et Alan se sont assez vite liés d'amitié en entrant à l'Epita, et en entendant parler du projet, ils se sont dit qu'ils le feraient ensemble. De plus, le fait que nous soyons dans la même classe facilitait grandement les choses (nous avons les mêmes horaires, etc…). Il fallait ensuite trouver un quatrième membre. Thomas était dans la classe et nous nous entendions assez bien avec lui. Il a finalement rejoint l'équipe. 
Pour finir, nous devions déterminer le chef de projet. Pour ça, rien de plus simple : nous sommes en démocratie, nous avons donc voté ! Thibault a créé un strawpoll sur internet, et tous les membres étaient candidats d'office. Chacun pouvait se défendre et argumenter sur sa capacité à diriger l'équipe. Théo a finalement été élu. 

	\section{Origine du projet}
	\paragraph{}
	La question de la nature du projet ne se posait même pas au sein du groupe, nous voulions faire un jeu. Nous devions travailler sur un projet en équipe et apprendre de nouvelles choses tout en y prenant plaisir, et c'est le jeu vidéo qui nous parlait le plus. C'est ainsi que nous avons pris la décision d'en réaliser un. Après il ne restait plus qu'à décider quel type de jeu nous voulions réaliser. Au début, Alan avait proposé de réaliser un RPG (Role Play Game), mais nous avons plus opté pour l'idée de Thibault : la DeathRun. Mais une DeathRun simple ne suffisait pas, nous voulions ajouter une touche plus personnelle sur le mode, alors nous nous sommes dit que nous allions mélanger le mode de jeu de la DeathRun et du RPG : il en ressort une DeathRun avec différents personnages qui possèdent chacun des attributs spécifiques et des bonus.
	
	\section{Les bénéfices pour le groupe}
	\paragraph{}
		Ce projet d'informatique possède plusieurs avantages. Tout d'abord nous allons travailler en équipe. En effet, nous avons un chef de projet et chacun est responsable ou suppléant dans un domaine de réalisation. Cela va nous permettre d'acquérir une expérience indispensable en tant que futurs ingénieurs. De plus, nous pourrons aborder de nouvelles notions tel que le développement réseau (site web, réseau multijoueur…) ou encore le modélisme, que nous n'aurions pas forcément vus en classe. Nous aurons aussi l'occasion de travailler sur de la POO (Programmation Orientée Objet), assez différente de la programmation fonctionnelle que nous avons majoritairement vue en cours jusque là, et qui est pourtant indispensable dans le domaine de la programmation.

	\section{Les membres}
		\subsection{Thibault THOMAS}
		\paragraph{}
		
J'ai passé un bac S et, étant intéressé par le domaine de l'informatique, j'ai décidé de postuler à l'\'EPITA. En arrivant dans cette école, je n'avais aucune notion de programmation. Je ne savais ni comment fonctionnait un langage ni ce qu'était un terminal !
A vrai dire, mon utilisation d'un ordinateur se résumait à YouTube et aux jeux vidéo. C'est donc en toute logique que l'idée de réaliser un jeux vidéo m'a vite plue ! \\
Question travail de groupe, mon expérience s'arrête aux exposés au collège et lycée et au TPE en $\textrm{1}^\textrm{ère}$, une expérience assez basique finalement. Mais je suis très enthousiaste à l'idée de réaliser un projet aussi sérieux. \\
		
			 Ce projet est une aubaine pour moi, c'est l'occasion de combler mon domaine de compétence. C'est d'ailleurs pour ça que je me suis proposé pour être responsable du développement du site internet, cette compétence est basique est indispensable en tant qu'informaticien. De plus, j'aurai l'occasion de travailler sur le code même du jeu (surtout de la POO à vrai dire) en tant que suppléant. Cette notion étant assez complexe à appréhender en tant que débutant, j'aurai l'occasion de me plonger pleinement dedans et ainsi de progresser un maximum. 
		
		\subsection{Alan GUERET}
		\paragraph{}
		
			En sortant d'une terminale Sciences de l'ingénieur, option ISN, j'ai déjà réalisé deux projets en groupe. Celui d'ISN se rapproche plus de ce que nous devons faire cette année dans la mesure où c'était un projet informatique. Il était cependant peu comparable à celui que l'on va devoir réaliser cette année en terme de difficulté et de charge de travaille. Nous étions, pour ce projet en groupe de trois, nous avons implémenté un gestionnaire de mot de passe en essayant de le rendre le plus sécurisé possible en python. Dans ce projet je m'occupais de la génération de mots de passe aléatoire et de créer la fenêtre pour saisir une nouvelle entrée (nom du site, identifiant, mot de passe).Pour le projet de Sciences de l'ingénieur, nous étions un groupe de 4 et nous avons réalisé un chariot qui porte la valise d'un utilisateur en le suivant dans un aéroport. J'ai aussi fais plusieurs autres projets avec mon père et mon frère comme une araignée imprimée en 3D et contrôlée par arduino ou un réveil avec un afficheur LCD.\\

Ce projet est très important à mon avis. Il nous permet d'avoir une première expérience du travail d'équipe. Ayant déjà travaillé sur des projets en équipe en terminal, un projet
informatique en python et un en science de l'ingénieur, je pourrais partager mon expérience avec mes collègues. Personnellement, je pense que le projet permet d'appliquer une grande partie des cours théoriques ce qui me permet être l'une des meilleures façon d'apprendre. La programmation est un domaine qui m'intéresse fortement, ce projet va me permettre de me rendre compte de ce qu'est ce métier, bien que j'en ai une vague idée. Ce projet me permettrait d'apprendre à réaliser un jeu ou un site web, ce qui pourrait m'être utile. Les soutenances vont m'aider à m'améliorer à l'oral et la qualité de mes présentations. 

		
		\subsection{Thomas PHILIP}
		\paragraph{}
			J'ai passé un BAC S spécialité mathématique, en vu de pouvoir faire des études en programmation. Au collège j'ai effectué un stage à l'ISART Digital m'initiant à la programmation et créant aussi une certaine passion pour ce domaine. Par la suite j'ai effectué plusieurs stages consistant pour la plupart de créer des jeux vidéo. Mon plus gros projet c'est déroulé sur un mois, dans ce projet j'ai été désigné chef de projet avec un camarade à la tête d'un groupe de trente personnes afin de créer un jeu sur Unreal Engine sans s'aider des différentes bibliothèques existantes (code source, graphisme, animation, son, etc …). Ces différentes expériences m'ont poussé à étudier l'informatique, et ainsi m'ont incité à rejoindre l'EPITA.
			
Personnellement ce projet peut m'apporter énormément, en effet étant suppléant dans graphisme et son je pourrais découvrir un aspect du jeu vidéo que l'on oublie souvent. Cela me permettra de m'initier à la modélisation 3D ainsi qu'au traitement de piste audio. Ensuite mon rôle de responsable dans la programmation me permettra d'améliorer mes compétences actuelles dans ce domaine ; ayant déjà utilisé Unity pour d'autres projets d'ampleurs différentes, j'espère aiguiser mes connaissances et découvrir de nouvelle manière d'approcher ce logiciel. Enfin j'espère pouvoir apporter de mon expérience à mon équipe tout en développant mes qualités d'expressions.

		\subsection{Théo DARONAT}
		\paragraph{}
		
			Au lycée, j'ai eu la chance de pouvoir aller dans la filière SI. En terminal j'ai pu vivre une expérience qui a l'air similaire à ce projet de S2, dans la forme du moins. Mes anciens camarades et moi avons créé différents groupes, le mien étant un groupe de 4, et nous devions sur un semestre réaliser un projet de notre choix. Nous avons décidé de créer un frigo connecté et autonome. Pour ce faire nous avons décidé que deux personnes s'occuperaient de la partie mesure (température, hygrométrie, etc...) et que deux autres personnes se chargeraient de la partie programmation. Je faisait partie du groupe orienté vers le développement et, n'ayant jamais programmé avant, mes professeurs m'ont conseillé d'utiliser "MIT APP INVENTOR 2", une application ressemblant fortement à scratch. Ce projet m'a permis de consolider mon choix d'étudier et de travailler dans l'informatique.\\
			
J'attends beaucoup de ce projet pour plusieurs raisons. J'espère apprendre beaucoup du travail de groupe et de l'organisation d'un tel projet. Ce projet va également me permettre de découvrir la façon de créer un jeu et ses contraintes, d'aborder la facette multijoueur d'un jeu vidéo, de penser et de créer sons, bruitages et ambiances sonores. De plus je m'occupe de la modélisation ce qui est, pour moi, un grand défi à relever. Mais surtout, ce projet est l'occasion de réaliser une promesse que je m'étais faite étant petit et de me rapprocher un peu plus du métier de développeur. J'espère prendre beaucoup de plaisir à réaliser ce projet.

\chapter{\'Etat de l'art}
	\paragraph{}
	Le type de jeu Death Run est en réalité un mode de jeu incorporé dans des jeux déjà existant. Le premier à avoir eu ce mode de jeu est le "jeu" Garry's Mod qui est un jeu "sandbox" (jeu bac à sable) où le principe est de créer des modes de jeu. Ce sont les utilisateurs qui font évoluer le jeu. Ce dernier, créé par Garry Newman, a été publié par la société Valve en 2006, il faudra attendre 2009 pour que le mode de jeu Death Run se popularise. Le principe du mode Death Run est plutôt simple, il y a deux équipes. La première est celle des coureurs, leur but est d'aller d'un bout à l'autre de la map, la seconde est composé généralement d'un seul joueur qui a pour objectif d'empêcher les coureurs de terminer la course, en activant des pièges. Ce mode de jeu possède plusieurs aspects attrayant, comme la compétitivité par exemple. En effet pour progresser dans ce mode, on doit pratiquer et acquérir des compétences de mécanique de jeu qui n'ont aucune limite : on a toujours une marge de progrès. De plus l'aspect teamplay et coopératif en multijoueur rend le jeu toujours plus amusant. Enfin, l'activité constante de la communauté pemret un renouvellement constant du mode (maps, pièges, etc...), ce qui règle le problème de la lassitude à long terme. En effet, les possibilités de création et d'innovation sont immenses : la seule limite est l'imagination.
	\paragraph{}
	On peut évoquer d'autres jeux populaires qui utilisent le mode de la Death Run, comme Counter-Strike. Ce dernier est à la base un mode de jeu de Half Life, un fps sorti en 1998 et développé par Valve. Il est sorti en tant que jeu officiel en 2000. Ce qui nous intéresse ici n'est pas tant le mode principal de Counter-Strike, mais plutôt l'aspect communautaire. En effet, Valve a mis à disposition des moyens de créer des serveurs publiques par et pour la communauté, et c'est ainsi qu'est né le DeathRun sur Counter-Strike.
	Il existe d'autres jeux populaires possédant ce mode jeu tel que Team Fortress 2 : un jeu de tir à la première personne multijoueur en ligne Développé par Valve Corporation qui est construit autour de deux équipes s'affrontant pour un objectif. On peut également cité le jeu le plus populaire depuis un an qui n'est nulle autre que Fornite, un jeu style Battle Royal possédant des pièges et un mode bac à sable, ce qui a permis la création d'un mode Death Run. Enfin, on peut évoquer le jeu Minecraft qui, grâce à la grande liberté qu'il nous offre, permet aux joueurs de créer de nombreux modes de jeu dont des Death Run.

\chapter{Description et mécanique du jeu}

	\section{Concept du jeu}
	\paragraph{}
		Le jeu permet au joueur d'incarner différents personnages avec différentes capacités. Une partie se déroule sur un parcours d'obstacle rempli de pièges multiples et différents. Le but d'une partie est d'arrivée au bout du parcours avant le temps imparti ; si plusieurs joueurs jouent ensemble sur une même partie, c'est le premier joueur qui arrive au bout du parcours qui fait gagner son équipe. Dans une partie il existe aussi un joueur qui incarne le "maitre du jeu" dont la tâche sera d'empêcher les autres joueurs d'arriver au bout du parcours.
	
Lors d'une partie, certains obstacles (activés ou non par le "maître du jeu") peuvent éliminer un joueur pendant un certain temps. Une fois le temps écoulé, le joueur peut réapparaître au dernier point de contrôle. Ces points de contrôle se trouvent tout au long du parcours et pour les activer, il suffit de les dépasser. La partie s'arrête si un joueur atteint la ligne d'arriver.

	\section{Mode Solo}
	\paragraph{}
		En mode solo, une série de niveaux vous seront proposés, à travers ces niveaux vous pourrez découvrir les différentes mécaniques du jeu. Les architectures des parcours proposés seront légèrement différentes des architectures du jeu en multijoueur, en effet, des monstres seront présent sur votre parcours et les pièges ne seront pas activés par un joueur. On pourra aussi trouver un niveau "Boss". Dans ce niveau, le joueur jouera sur une course normale, mais avec un "Boss" en plus, qui sera sur la course et lancera des projectiles sur le joueur. Pour vaincre le boss, il faut terminer la course. Les intelligences artificielles devront suivre le joueur afin de l'attaquer ou activer les pièges de manière aléatoire.

	\section{Mode Multijoueur}
	\paragraph{}
		Le "maitre du jeu" aura une vision globale du terrain avec ces pièges et les joueurs se trouvant dessus. Il pourra aussi attribuer à toute l'équipe de joueurs se trouvant sur le parcours des malus qu'il aura préalablement choisi avant la partie. Le "maitre du jeu" gagne si les joueurs n'ont plus de vie.
		
	\section{Terrain et pièges}
	\paragraph{}
		Il y aura différent terrains qui auront des ambiances variantes. 
		Il y aura différents pièges comme des pics sortant du sol, des murs poussant les joueurs dans le vide, des flèches, un boule géante, etc...

	\section{Personnage}
	\paragraph{}
	
		Différent personnage seront jouable dès le début du jeu : 
	
		\subsection{Lancelourd}
		\paragraph{}
			"LanceLourd" est un chevalier sur sa monture possédant une capacité active qui est un mouvement rapide vers l'avant sans perte altitude. Cette capacité permet aussi de projeter d'autres joueurs au contact "physique". "LanceLourd" possède aussi une compétence passive; le personnage est représenté sur sa monture, si "LanceLourd" est éliminé une première fois il est projeté sur la surface non dangereuse la plus proche mais perd son fidèle serviteur.
		
		\subsection{Gandoulf}
		\paragraph{}
			"Gandoulf" est un magicien possédant une capacité active permettant d'échanger sa position avec un autre joueur de manière aléatoire dans une zone limitée. "Gandoulf" possède aussi une compétence passive qui applique un bonus aléatoire à toute son équipe tout les x secondes pour une durée de x secondes.
		
		\subsection{Alibouba}
		\paragraph{}
			"Alibouba" est un voleur qui possède une capacité active permettant d'obtenir la capacité d'un autre joueur sur le terrain; cette capacité commence son temps de recharge une fois que la capacité volée est utilisée. Si jamais "Alibouba" est éliminé et réapparait s'il avait volé une capacité avant l'élimination, celle-ci est perdu. Il possède aussi une capacité passive lui permettant d'effectuer des doubles sauts.
		
		\subsection{Jusé}
		\paragraph{}
			"Jusé" est un moine possédant une capacité active permettant de retirer un statut aléatoire qui lui est appliqué en le donnant à un autre joueur. Lors de l'obtention du statut par l'autre joueur les caractéristiques de temps du statut ne changent pas (par exemple, si l'effet du statut devait encore perdurer 17 seconde sur "Jusé", le statut obtenu par l'autre joueur serait actif 17 secondes).  "Jusé" possède aussi une capacité passive qui lui permet de léviter à une certaine hauteur pendant un temps limité.
					
\chapter{Découpage du projet} % technologique et méthodologique

	\section{Les 4 tâches principales}
	
		\subsection{Multi}
		\paragraph{}
Ceux qui s'occupent de cette partie devront implémenter en C\# le 
fait de pouvoir jouer à notre jeu à plusieurs. Pour se faire nous
utiliserons photon qui est une librairie C\#. 

		\subsection{Site Internet}
		\paragraph{}
Pour créer un site internet nous utiliserons un framework tel que
ruby on rail ou angular js. Nous y présenterons et mettrons à
disposition notre jeu. Nous nous aiderons d'autres sites web et 
de cours pour pouvoir le réaliser.

		\subsection{Graphisme et Son}
		\paragraph{}
		Les personnes qui s'occupent de cette partie seront en charge de toute la partie visuelle du jeu. Ils utiliseront blender pour la partie visuelle. Pour la parie audio, ils utiliseront des
des instruments diverses (guitare, batterie...) ou simplement
leurs voies. Ils utiliseront aussi des logiciels de montage.\\

		\subsection{Code Unity}
		\paragraph{}
			Cette partie a pour but de réaliser toute la partie logique du jeu. En effet, ceux qui feront cette partie seront en charge d'implémenter la physique du jeu, les menus, les gestion de 
tous les éléments qui doivent apparaître à l'écran... Nous
utiliserons le C\# avec la plateforme Unity.\\

	\section{Répartition des tâches}	
	
	\begin{tabular}{|*{5}{c|}}
	\hline
		& Réseaux - Multi & Site internet & Graphisme et son & Code unity \\
		\hline
		Thomas PHILIP & - & - & Suppléant & Responsable \\
		\hline
		Alan GUERET & Responsable & Suppléant & - & - \\
		\hline
		Thibault THOMAS & - & Responsable & - & Suppléant \\
		\hline
		Théo DARONAT & Suppléant & - & Responsable & - \\
		\hline
	
	\end{tabular}
	\\
	\\
	
	\section{Objectifs de rendu}
	
	\begin{tabular}{|*{4}{c|}}
	\hline
	& \multicolumn{3}{|c|}{Deadline}\\
	\hline
	Tâche & du 9 au 13 mars & du 20 au 24 avril & du 28 mai au 10 juin \\
	\hline
	Interface jeu & & x & \\
	\hline
	Map multi (modélisation) & & x & \\
	\hline
	synchronisation multi & x & & \\
	\hline
	Map solo (modélistion) & & x & \\
	\hline
	effet sonore & & & x \\
	\hline
	ambiance sonore & & & x \\
	\hline
	site web & x & & \\
	\hline 
	physique des objets & x & & \\
	\hline 
	level design & & & x \\
	\hline 
	IA des ennemis & & x & \\
	\hline
	IA du boss & & & x \\
	\hline
	
	\end{tabular}
\chapter{Conclusion}
site intéressant cahier des charges : \url{http://forum.mathematex.net/latex-f6/reduire-au-maximum-les-marges-t7247.html}\\

\end{document}