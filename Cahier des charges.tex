\documentclass[12pt]{report}
\title{\large Studio Run \\ \normalsize Dungeon Race}
\date{}
\author{Thibault THOMAS \\ Thomas PHILIP \\ Alan GUERET \\ Théo DARONAT}
\begin{document}
\maketitle
\tableofcontents
\section{Introduction}
Il faut rédiger une introduction d'au minimum une page, c'est la merde. Après les pages ont l'air d'être petite. Les gars, c'est le feu. Une ligne sur TeXmaker équivaut à une ligne sur le doc pdf ! Ah, nan finalement...
\section{Plan du projet}
	\subsection{Origine de l'étude}
	à faire
	\subsection{Objet de l'étude}
		utilisation de unity, git, photon, blender.
	\subsection{\'
	Le type de jeu Death Run est en réalité un mode de jeu incorporé dans des jeux déjà existant. Le premier à avoir eu ce mode de jeu est le "jeu" Garry's Mod qui est un jeu "sandbox" (jeu bacEtat de l'art}	
	\paragraph{} à sable) où le principe est de créer des modes de jeu. Ce sont les utilisateurs qui font évoluer le jeu. Ce dernier, créé par Garry Newman, a été publié par la société Valve en 2006, il faudra attendre 2009 pour que le mode de jeu Death Run se popularise. Le principe du mode Death Run est plutôt simple, il y a deux équipes. La première est celle des coureurs, leur but est d'aller d'un bout à l'autre de la map, la seconde est composé généralement d'un seul joueur qui a pour objectif d'empêcher les coureurs de terminer la course, en activant des pièges. Ce mode de jeu possède plusieurs aspects attrayant, comme la compétitivité par exemple. En effet pour progresser dans ce mode, on doit pratiquer et acquérir des compétences de mécanique de jeu qui n'ont aucune limite : on a toujours une marge de progrès. De plus l'aspect teamplay et coopératif en multijoueur rend le jeu toujours plus amusant. Enfin, l'activité constante de la communauté pemret un renouvellement constant du mode (maps, pièges, etc...), ce qui règle le problème de la lassitude à long terme. En effet, les possibilités de création et d'innovation sont immenses : la seule limite est l'imagination.
	\paragraph{}
	On peut évoquer d'autres jeux populaires qui utilisent le mode de la Death Run, comme Counter-Strike. Ce dernier est à la base un mode de jeu de Half Life, un fps sorti en 1998 et développé par Valve. Il est sorti en tant que jeu officiel en 2000. Ce qui nous intéresse ici n'est pas tant le mode principal de Counter-Strike, mais plutôt l'aspect communautaire. En effet, Valve a mis à disposition des moyens de créer des serveurs publiques par et pour la communauté, et c'est ainsi qu'est né le DeathRun sur Counter-Strike.
	Il existe d'autres jeux populaires possédant ce mode jeu tel que Team Fortress 2 : un jeu de tir à la première personne multijoueur en ligne Développé par Valve Corporation qui est construit autour de deux équipes s'affrontant pour un objectif. On peut également cité le jeu le plus populaire depuis un an qui n'est nulle autre que Fornite, un jeu style Battle Royal possédant des pièges et un mode bac à sable, ce qui a permis la création d'un mode Death Run. Enfin, on peut évoquer le jeu Minecraft qui, grâce à la grande liberté qu'il nous offre, permet aux joueurs de créer de nombreux modes de jeu dont des Death Run.
	\subsection{Découpage du projet} % technologique et méthodologique
	trois à quatre pages\\
	\\
	répartition des tâches \\
	\begin{tabular}{|c|c|c|c|c|}
	\hline
		& Réseaux - Multi & Site internet & Graphisme et son & Code unity \\
		\hline
		Thomas PHILIP & - & - & Suppléant & Responsable \\
		\hline
		Alan GUERET & Responsable & Suppléant & - & - \\
		\hline
		Thibault THOMAS & - & Responsable & - & Suppléant \\
		\hline
		Théo DARONAT & Suppléant & - & Responsable & - \\
		\hline
	
	
	\end{tabular}
\section{Conclusion}
\end{document}